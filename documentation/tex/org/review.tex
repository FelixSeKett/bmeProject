\textbf{Organisation:}
Die Gruppe "Edwards Biotope", bestehend aus 10 Leuten, hat sich vor Beginn des Wintersemesters 2018/19 zusammengefunden, um sich eine Projektidee zu überlegen. Es soll ein Strategie Videospiel entstehen. Das Konzept wurde durch interne Projektpräsentationen und Diskussionen fertig ausgearbeitet und in Form eines Projektpapier formuliert. Mit dem fertigen Konzept wurde bei der Themenvorstellung für die Interdisziplinären Projekte vor den Studenten und betreuenden Professoren unseres Semesters für die Idee geworben. Die finale Gruppeneinteilung und der betreuende Professor Schedel waren somit offiziell und endgültig festgelegt. Zu Beginn fanden viele Besprechungen statt, in denen sich geeinigt werden musste, um was es in dem Spiel gehen soll, wie der Stil des Spiels sein soll, welche Frameworks zur Ausarbeitung und Realisierung benutzt werden und wie die Arbeit mit den bevorstehenden Aufgaben organisiert werden soll. Im Team wurde beschlossen, nach der agilen Scrum Methode zu entwickeln (Scrum Master: Sebastian Beck, Pia Korndörfer / Product Owner: Felix Baumgarten, Robert Sabo). Die Infrastruktur zur Projektorganisation sowie Kommunikation wurde mit Hilfe von GitKraken realisiert (Kombination aus GitHub und einfache Projekt-Management Lösung). Damit entsprechend wurde eine zentrale Lösung gefunden, in der man zum einen die verschiedenen Software-Stände und zugehörige Branches verwalten kann, aber auch alle Aufgaben innerhalb des Projekts mittels einem sogenannten „Glo-Board“ dokumentieren und zuweisen kann. Die Marke "Edwards Biotope" wurde als originale IP etabliert und profiliert.

\textbf{Game Design:}
Zunächst wurde eine Detailrecherche zu den Game Design-Anforderungen gemacht. Also die Spielmechanik bereits bestehender und erfolgreicher Trading Card Games wurde analysiert und bewertet. Aber auch auf visueller Basis wurden Moodboards zum Spieleuniversum erstellt. Ein komplett neues, originales Gameplay wurde entwickelt und mit Spielabläufen und Spielregeln festgehalten. Um die Spielmechanik besser nachzuvollziehen, wurde ein analoger Prototyp des Spiels produziert. So konnte die Spielmechanik, vor allem am Anfang unseres Projektes, verstanden und gegebenenfalls verbessert und mit manuellen Tests optimiert werden. Aber auch die Story und das Spieluniversum wurde durch Ausarbeitung und internen Präsentationen geformt und gestaltet. Dabei war es wichtig Spielinterne Mechaniken mit einer Story zu verknüpfen. Im Zuge dessen wurde sich auch auf eine Namensfindung des Produkts "Edwards Biotope" geeinigt.

\textbf{Ästhetik:}
Viele Grundlegende ästhetische Vorgaben wurden besprochen und festgelegt. Der Stil der Gesamt-Ästhetik und Tonalität, ein Corporate Design mit Logo, Farbpalette und verwendeten Schriftarten.
Aber auch grundsätzliche Bedienelemente wie Icons und Buttons, aber auch Checkboxen und Eingabefelder wurde angefertigt. Um einen einheitlichen Zeichenstil der einzelnen Illustrationen für die Spielkarten beizubehalten, wurde auch ein Work Flow der Illustrationen definiert. Durch Aufteilung in einzelne Abschnitte (Skizzen, Outlines, Colorblocking, Highlights und Shading, Hintergrund) wurde eine einheitliche Illustration gewährleistet. Diese wurde zum letzten Semesterende bis zu etwa 50 Prozent fertiggestellt. Sowohl zum Duell-Bildschirm, als auch zum Kartendeckeditor, wurden mehrere digitale Prototypen als Vorlage erstellt. Kleinere Mechaniken konnten so getestet werden. Diese wurden mit dem Prototyping Tool Figma realisiert.

\textbf{Software-Entwicklung:}
Mit ausreichender Recherchearbeit zur idealen Programmiersprache und dazugehörigem Framework, wurde sich gemeinschaftlich auf Java und LibGDX geeignet (für unser Projekt ausreichender Umfang und nicht zu komplex). Zusätzlich wurde jedes Mitglied mit entsprechender Entwicklungsumgebung ausgestattet und ein dezentrales Repository mit Versionsverwaltung eingerichtet. Das favorisiertes Werkzeug zur Java-Entwicklung ist die Entwicklungsumgebung IntelliJ IDEA.
Um nun bei steigender Komplexität sicherstellen zu können, dass auch mit neuen Implementationen alte Komponenten lauffähig bleiben, wurden automatisierte Softwaretests mit dem Framework JUNit verwendet. Recherchearbeit und Tests zu Multiplayer-Funktionen wurde mit Socket.io und NodeJS gehandhabt. Nach ersten Implementierungen eines Codestamms, sowie einfachen Erstellungen von Klassendiagrammen und witeren UML-Diagrammen, wurde nach und nach die Architektur der Software spezifiziert und ausgearbeitet. Die ersten Codes wurden schon geschrieben, um eine Basis der noch komplexer werdenden Software zu legen.

Beteiligte Personen: Felix Baumgarten, Sebastian Beck, Pamela Schättin, Manuela Mosandl, Cigdem Bozyigit , Gabriel Veiz, Pia Korndörfer, Philadelphia Gauß, Robert Sabo, Daniel Scharrer (nur im ersten Projektabschnitt dabei)