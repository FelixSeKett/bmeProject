\subsection{Planungsgremium}
Aufgabe des Planungsgremiums war die Planung des Präsentationstages. 
Teil davon war die Organisation des Ablaufes sowie der Örtlichkeit der Präsentationen, als auch die Planung der daran anschließenden Abschlussfeier. 

\subsection{Werbeplakat und technisches Plakat}
Zur Projektpräsentation sollten ein Werbeplakat und ein technisches Plakat erstellt werden. Das Werbeplakat vom letzten Semester wurde mit kleinen Veränderungen übernommen. Letztendlich musste nur ein Plakat zur Erklärung und Visualisierung aller technischer Komponenten hergestellt werden. Diese wurden mit InDesign in einem DIN A1 Querformat realisiert. Dabei war es wichtig im selben Stil wie das Werbeplakat zu gestalten. Es sollten minimalistische Formen und Icons genutzt werden. Bezüge sollte nur durch wenig Text erklärt werden. Bezüge und allgemeine technische Funktionen unseres Videospiels sollen verstanden und erkannt werden.

\subsection{Projektpräsentation und Demo}
Um nun unseren aktuellen Stand von "Edwards Biotope" an dem Projektpräsentationstag vorzustellen, musste eine Präsentation her. Diese wurde von Robert Sabo und Gabriel Veiz strukturiert und erstellt. Der Fokus lag dabei eher auf das Produkt und weniger auf das Projekt an sich. Es sollte klar gezeigt werden, was "Edward Biotope" ist, was es kann. Präsentiert haben Pia Korndörfer, Gabriel Veiz, Sebastian Beck und Felix Baumgarten. Neben allgemeinen Informationen und visuellen Erklärungen, wurden auch Spielregeln anhand des fertigen Spiel gezeigt. Bei der darauffolgenden offenen Demo konnte "Edwards Biotope" mit Studenten und Gästen getestet und gespielt werden.