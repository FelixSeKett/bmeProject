\subsection{Feinschliff: Fokus auf Stellungsspiel}
Bei Praxistests und in der Live Demo des letzten Semesters hat sich herausgestellt: Interessantestes Merkmal unserer Spielmechanik ist das Stellungsspiel der durch die Karten repräsentierten Spielobjekte.
Wo sich vergleichbare Trading Card Games stärker auf Zahlenschieberei konzentrieren (Verrechnen von Statuswerten, Sammeln von Zählmarken, Jonglieren mit Würfelergebnissen) orientiert sich unser Produkt am Prinzip des Brettspielklassikers Schach, dessen emergentes Potenzial ausschließlich aus der Vielfalt der Anordnungsmöglichkeiten seiner Figuren erwächst.
Die positive Resonanz zu diesem Umstand veranlasste Robert Sabo und Felix Baumgarten dazu, den Spielregeln einen letzten Feinschliff zu verpassen; im Zuge dessen die wenigen noch existenten Zahlen (Stärkewerte im oberen rechten Eck jeder Karte) über Bord zu werfen und das Stellungsspiel vollständig in den Mittelpunkt zu rücken. Zwar finden für konfrontative Auseinandersetzungen unter den Spielkarten noch immer Schadensberechnungen statt, doch basieren diese nicht mehr auf dem Vergleich individueller Statuswerte, sondern auf Kartentypen-abhängigen Eigenschaften:
\begin{itemize}
\item Jede Karte verfügt abhängig von ihrem Kartentyp über eine feste Anzahl an Trefferpunkten:
\begin{itemize}
\item Quartiere: 3 Trefferpunkte
\item Kreaturen: 2 Trefferpunkte
\item Phänomene: 1 Trefferpunkt
\end{itemize}
\item Ein Angriff reduziert die Trefferpunkte des Zieles stets um 1
\end{itemize}
Auf diese Weise wird das Regelwerk reduziert, die Kartengestaltung entschlackt und der Spielablauf vereinfacht – und gleichzeitig das Hauptaugenmerk unseres Gameplays unterstrichen.
Beteiligte Personen: Robert Sabo, Felix Baumgarten
