\subsection{Feinschliff: Fokus auf Stellungsspiel}
Bei Praxistests und in der Live Demo des letzten Semesters hat sich herausgestellt: Interessantestes Merkmal unserer Spielmechanik ist das Stellungsspiel der durch die Karten repräsentierten Spielobjekte.
Wo sich vergleichbare Trading Card Games stärker auf Zahlenschieberei konzentrieren (Verrechnen von Statuswerten, Sammeln von Zählmarken, Jonglieren mit Würfelergebnissen) orientiert sich unser Produkt am Prinzip des Brettspielklassikers Schach, dessen emergentes Potenzial ausschließlich aus der Vielfalt der Anordnungsmöglichkeiten seiner Figuren erwächst.
Die positive Resonanz zu diesem Umstand veranlasste Robert Sabo und Felix Baumgarten dazu, den Spielregeln einen letzten Feinschliff zu verpassen; im Zuge dessen die wenigen noch existenten Zahlen (Stärkewerte im oberen rechten Eck jeder Karte) über Bord zu werfen und das Stellungsspiel vollständig in den Mittelpunkt zu rücken. Zwar finden für konfrontative Auseinandersetzungen unter den Spielkarten noch immer Schadensberechnungen statt, doch basieren diese nicht mehr auf dem Vergleich individueller Statuswerte, sondern auf Kartentypen-abhängigen Eigenschaften:
\begin{itemize}
\item Jede Karte verfügt abhängig von ihrem Kartentyp über eine feste Anzahl an Trefferpunkten:
\begin{itemize}
\item Quartiere: 3 Trefferpunkte
\item Kreaturen: 2 Trefferpunkte
\item Phänomene: 1 Trefferpunkt
\end{itemize}
\item Ein Angriff reduziert die Trefferpunkte des Zieles stets um 1
\end{itemize}
Auf diese Weise wird das Regelwerk reduziert, die Kartengestaltung entschlackt und der Spielablauf vereinfacht – und gleichzeitig das Hauptaugenmerk unseres Gameplays unterstrichen.

\subsection{Die Spielmechanik der Strömung}
Im Zuge der Demoveranstaltung im vorherigen Semester, konnten Studenten, Professoren und weitere Besucher, das Spiel am analogen Prototypen testen. Viele User hatten Erfahrung mit Trading Card Games und konnten uns neben gutem Feedback auch Verbesserungsvorschläge zum Spiel geben. Der generelle Konsens fand den Fokus auf das Stellungsspiel einen interessanten Ansatz, den nicht viele Card Games berücksichtigen. Darauffolgend, haben sich Felix Baumgarten und Robert Sabo an die Spielmechanik gesetzt und ein weiteres Element des Stellungsspiels implementiert. Die Strömungsrichtung. Mann kann entweder im Uhrzeigersinn, oder gegen den Uhrzeigersinn angreifen. Ein Angriff wird jetzt immer in Strömungsrichtung ausgeführt. Aber auch Quartiere verschieben ihre Kreaturen und Phänomene in diese Richtung. Visuell wird das ganze über einen Wasserstrom angezeigt, der hinter den Spielkarten in die jeweilige Richtung rotiert. Damit ist auch diese Funktion innerhalb des Spieluniversums(Unterwasserwelt) kreativ umgesetzt und sie ist auf dem ersten Blick direkt sichtbar. Die Strömungsrichtung kann immer dann über einen Button manipuliert werden, wenn auch der Kompass verstellbar ist. Diese beiden Aktionen fallen unter dem Begriff "Spielfeld vorbereiten". Wenn diese nach Wunsch vorbereitet wurden, können Spielkarten gelegt und Farbzonen aktiviert werden. Das Element der Strömungsrichtung fördert wie der Kompass das Stellungsspiel des Kartenspiels.