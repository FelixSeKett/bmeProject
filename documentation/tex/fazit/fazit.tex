Am Ende der zwei Projektsemester steht ein Produkt, mit dem das Team zufrieden sein kann. Insbesondere das Spielprinzip und die audiovisuelle Präsentation wurden intern und extern bei verschiedenen Gelegenheiten als überraschend gelungen hervorgehoben, was dem im Projektpapier festgehaltenen ursprünglichen Ansatz entspricht, technologische Innovationen oder wissenschaftliche Werte zugunsten der Annäherung an eine vertriebsreife Produktqualität zu vernachlässigen.
Im Laufe der vergangenen Monate haben wir gelernt, dass persönliches Interesse für die Materie in immenser Leistungsbereitschaft münden kann, aber auch, dass Arbeitsagilität und damit letzten Endes auch -effizienz mit Teamgröße, Kommunikation und persönlichem Befinden einhergehen. Obwohl wir es schafften, regelmäßige Treffen abzuhalten und engagiert in das erste Projektsemester zu starten, stellte sich gegen Mitte eine Motivationsflaute ein, die erst dann wieder weichen sollte, als erste prestigeträchtige Spielfunktionen erfolgreich und interaktiv visualisiert waren.
Schlussendlich war klar, dass das ursprünglich angestrebte Produkt mit seinen komplexen Features viel zu aufwändig sein würde, um es nach den gegebenen Rahmenbedingungen zu realisieren. Daher machten wir Abstriche im Gameplay und damit im Workload der Software-Entwicklung. So schafften wir es schließlich, die Essenz unseres Spieles – das strategische Spielduell – ansehnlich und lauffähig zu inszenieren und so unsere wichtigsten Ziele zu erfüllen.
