\subsection{Musik und Soundeffekte}
Die Musik soll für eine bestimmte Atmosphäre sorgen, im Fall von Edwards Biotope handelt es sich um eine comichafte, bunte Unterwasserwelt mit einer leicht düsteren Note. Die Aufgabe lag außerdem darin, das Gefühl zu vermitteln, sich direkt am Handlungsort zu befinden. 
Während die Musikproduktion im Allgemeinen schon hohe Anforderungen mit sich bringt, erhöhen sich diese speziell im Bereich der Spieleentwicklung noch um ein Vielfaches. Vor allem hat dies mit der Konzentration des Spielers zu tun: Die Musik muss eine Balance zwischen Unterhaltung und Ablenkung finden. Das Musikstück muss dabei relativ ruhig sein, damit sich der Spieler auf das Spiel konzentrieren kann, darf aber auch nicht zu monoton sein, damit es den Spieler nicht langweilt oder gar nervt. 
Um all das zu berücksichtigen, wurden zuerst einige Ideen gesammelt. Dann wurde festgelegt, dass die Grundstimmung durch Unterwassergeräusche und Effekte repräsentiert wird. Solche Geräusche wie “Gluckern”, das Wellenbrausen, die U-Bootgeräusche, das Brüllen eines Motors, einem Sonar, das Atmen in einer Unterwassermaske, der Krach von Unterwasserexplosionen und die Kollisionen von Metallgegenständen im Wasser, wurden in einem Track mit Hilfe der Digital Audio Workstation (DAW) Ableton Live 10 gemischt. Um die Tiefe des Sounds zu erreichen, wurden verschiedene Filter wie Hi/Low-Pass, Compressor, Equalizer und zahlreiche Effekte wie Phaser, Flanger, Reverb, Delay u.a. benutzt. So entstand die atmosphärische Basis für den Track. 
Für die gute Laune sorgen Lieder aus den Jahren um 1920. Es wurden Lieder herausgesucht, die beim Hören im Spieler ein gutes Gefühl erzeugen und positive Stimmung verbreiten sollen und, wenn auch entfernt, im Zusammenhang mit Wasser und Meer stehen. So fiel die Wahl auf hawaiianisch anmutende Musik und auf deutsche Seemannslieder in klassischer Interpretation. Die Lieder wurden zusammen mit den Hintergrundgeräuschen in der DAW von Audacity mit Hilfe der Effekte Fade In/Out kombiniert. 
