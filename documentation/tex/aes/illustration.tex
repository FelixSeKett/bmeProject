\subsection{Illustrationen}
Da dieses Kartenspiel stark im Einklang mit der dahinter stehenden Geschichte und der umgebenden Atmosphäre funktioniert, hatte die Fertigstellung der restlichen Kartenillustrationen hohe Priorität. Die Kolorierung sowie die Licht- und Schattengebung der ausstehenden Illustrationen brachten schließlich 46 individuelle Kartenmotive hervor.

\subsubsection{Spielkarten Outlines}
Um die Outlines aller Illustrationen herzustellen wurde Adobe Photoshop verwendet. Die vorher angefertigten Skizzen von Pamela Schättin und Daniel Scharrer, wurden gescannt und als Vorlage zur digitalen Ausarbeitung benutzt. Mit einem Canvasformat von 850 x 560 px wurde mit einer Linienstärke von 4 px die vorher positionierte Vorlage digital nachgezeichnet. Kleinere Änderungen bzw. Verbesserungen wurden dabei vorgenommen. Die Datei konnte nun als .psd-Datei abgespeichert werden und dem nächsten Prozess, also dem Colorblocking, zur Verfügung gestellt werden.

\subsubsection{Karten kolorieren}
Im 5. Semester der Projektarbeit wurden noch nicht alle Karten koloriert. Es waren noch 35 Karten, die keine Farbgebung erhalten haben. Die Kolorierung der Karten erfolgte auf einem MacBook Pro, mit der Software Adobe Photoshop CC 2019 und einem Grafiktablett von Bamboo. Um das einheitliche Gesamtbild sicherzustellen, wurden die Farben aus der Farbtabelle aus dem 5. Semester ausgewählt. Bei der Kolorierung wurde darauf geachtet, dass jede Karten einen ausgewogenen Komplementärkontrast hat. Zusätzlich benötigt jede Illustration mindestens drei Farben. Die Figuren müssen sich auch sichtbar von dem Hintergrund abheben. 
Entsprechende Materialien finden sich im Anhang unter "Karten\_Kolorieren".

\subsubsection{Lichtgebung und Schattierung}
Wie auch im letzten Semester gehörte das Setzen von Licht und Schatten zum letzten Schritt des Illustrationsdurchlaufes, welcher mit der Absicht der Erhaltung eines einheitlichen Illustrationsstiles erstellt wurde. Dabei unterteilte sich diese Aufgabe in die Nachbesserung der eingesetzten Farben innerhalb der geschlossenen Bereiche der Illustrationen sowie der eigentlichen Setzung von Schatten und Licht. Für diese Tätigkeiten wurde das Programm Procreate verwendet.

\subsubsection{Spielkarten Hintergründe}
Um die Hintergrüde aller Illustrationen herzustellen wurde Adobe Photoshop verwendet. Farblich wurde sich an der vorher festgelegt Farbpalette gehalten. Die vorher angefertigten Illustrationen (mit Outline, Colorblocking, Highlights und Shading) sind soweit fertig. Diese besitzen allerdings noch kein Hintergrund. Um diese Kreaturen, Quartiere und Phänomene nun noch in eine passende Umgebung zu versetzen, muss eine passende Unterwasserwelt her. Diese stellt in der Dimensionalen Wahrnehmung die hinterste Ebene dar. Das sich das Objekt(Kreatur, Quartier, Phänomen) visuell noch viel mehr vom Hintergrund absetzt, wurde zum Hintergrund ein minimalistischer Zeichenstil genutz(also ohne Outlines). Kleine kreative Spielereien, um die Spielkarten individueller zu machen, wurden hinzugefügt. Die fertige Illustration steht.

\subsubsection{Spielkartenrückseite Assets}
Da im Spiel auch Spielkarten verdeckt liegen sollen, sprich im Kartendeckstapel oder die Handkarten des Gegners, müssen auch Assets für die Rückseite einer Karte vorliegen. Um die Assets herzustellen wurde Adobe Photoshop verwendet. Diese soll visuell eindeutig und schnell erkennbar sein, da viele verschiedene bunte Farben auf der Kartenvorderseite (also Illustrationen) benutzt wurden. Um die Kartenrückseite davon abzuheben wurden wenige Farben und große Flächen benutzt. Zusätzlich wurde das Logo genutzt um eine eindeutige Zuweisung unserer IP herzustellen.

\subsubsection{Hauptmenü-Bildschirm Assets}
Um nicht direkt nachdem das Spiel gestartet wird in den Duell-Bildschirm zu kommen, wurde sich noch ein Hauptmenü-Bildschirm überlegt. Robert Sabo und Pamela Schättin kümmerten sich um die jeweiligen Assets die der Hauptmenü-Bildschirm beinhalten soll. Um die Assets herzustellen wurde Adobe Photoshop verwendet. Um einheitlich zu wirken, wurde der Illustrationsstil der Spielkarten hergenommen. Farblich wurde sich an der vorher festgelegt Farbpalette gehalten. Die Assets spiegeln das Spieluniversum, nämlich die Unterwasserwelt, visuell wieder. Es soll das Spieluniversum "Edwards Biotope" leicht "anteasern", und einen gewissen Eindruck des Spiels vermitteln. Das Background-Asset wurde von Robert Sabo entworfen und zeigt die Unterwasserwelt aus einem U-Boot heraus. links oben ist das Logo des Spiels zu sehen. Die Buttons wurden von Pamela Schättin erstellt. Die Buttons "Duell starten", "Deckeditor" und "Einstellungen" haben jeweils immer ein passendes Icon aus der Unterwasserwelt. Ausgegraute Buttons sollen noch nicht auswählbar und "klickbar" sein.

\subsubsection{Duell-Bildschrim-Assets}
Robert Sabo kümmerte sich um die jeweiligen Assets die der Duell-Bildschirm beinhalten soll. Um die Assets herzustellen wurde Adobe Photoshop verwendet. Um einheitlich zu wirken, wurde der Illustrationsstil der Spielkarten hergenommen. Farblich wurde sich an der vorher festgelegt Farbpalette gehalten. Die Assets spiegeln das Spieluniversum, nämlich die Unterwasserwelt, visuell wieder. Dabei ergibt sich eine gewisse Stimmung, die erreicht werden soll. Der Aufbau des Duell-Screens ist durch Skizzen und aus dem vorher angefertigten analogen und digitalen Prototyp entstanden. 
Nun zum Interface Design. Grade bei so einer komplexen Spielmechanik, ist es wichtig, das der User visuelle Eindrücke der gleichen Funktion zuordnen kann. Hierbei wurden viele grundlegende Gestaltungsgesetze für Interfaces beachtet und angewendet (Gesetz der Nähe, Gesetz der Gleichheit, Gesetz der Konstanz,...). Es soll auf dem Spielfeld klar ersichtlich sein, wo die Spielkarten abgelegt werden können, wo sich die jeweiligen Spielkartentypen wie z.B. Quartiere befinden, wo sich sowohl mein Deck, Friedhof und Handkarten, sowie die des Gegners befinden. Aber auch Interaktionsmöglichkeiten sollen auch auf dem ersten Blick ersichtlich sein. Dabei ist es wichtig das Spielfeld visuell von den Interaktionsmöglichkeiten wie Buttons und Anzeigefelder zu trennen. Das ist mit einem drei Spaltensystem gelöst worden. Die linke Spalte gibt Informationen über "Wer ist an der Reihe", einen Interaktionshinweis der Buttons und welche Karte ausgewählt ist wieder. Die rechte Spalte zeigt alle Buttons an, die mit dem Spielfeld interagieren und diese auch manipulieren können. Und die mittlere Spalte zeigt das tatsächliche Spielfeld, wo das Duell stattfindet, an. Um ein passendes Asset herzustellen müssen auch noch kommende Prozesse berücksichtigt werden. Sprich wie können diese Assets mit Hilfe der von IntelliJ IDEA vorhandenen Manipulationen, animiert werden. Rotieren, Skalieren, die Opacity ändern, Geschwindigkeit animieren etc. können später angewendet werden, um einen gewissen Interaktionsfeedback zu bekommen.

\subsubsection{Karten exportieren}
In Adobe InDesign musste der Rahmen der verschiedenen Kartentypen, mit den Kartenillustrationen und dem Kartennamen aus einer CSV Datei zusammengefügt werden. 
Der Rahmen musste noch einmal überarbeitet werden, da keine Effekte mehr auf der Karte ausgezeichnet werden. Deshalb wurde das Textfeld für die Effekte aus dem Rahmen mit Photoshop entfernt. Nach der Zusammenführung InDesign, entsteht ein PDF, aus diesem wurde mit dem Adobe Acrobat Reader Pro einzelne PNGs exportiert. Die Benennung der einzelnen Karten erfolgte aufgrund einer ID, so lässt sich die Karten im Sourcecode durch die ID zusammenführen.
Entsprechende Materialien finden sich im Anhang unter "Zusammenführung".



